\documentclass{report}
\usepackage{polski}
\usepackage{amssymb}
\usepackage{amsthm}
\usepackage{fullpage}
\usepackage{amsmath}
\usepackage{graphicx}
\usepackage[utf8]{inputenc}
\usepackage{pdfpages}
\usepackage{xcolor}
\usepackage{verbatim}
\usepackage{array}% http://ctan.org/pkg/array
\newtheorem{lemma}{Lemma}
\newtheorem{sublemma}{Lemma}[lemma]
\begin{document}
\newpage
Istnienie co najmniej jednego pierwiastka jest oczywiste -- $(3, 4, 5)$
jest trójką pitagorejską, więc dla $x=2$ mamy miejsce zerowe.
Szukam takich $x$, że $f(x) = 0$, czyli:
$$ 3^x + 4^x = 5^x $$
$$ (\frac{3}{5})^x + (\frac{4}{5})^x  = 1 $$
\noindent Zauważmy, że $(\frac{3}{5})^x$ i $(\frac{4}{5})^x$ są ściśle malejące
więc ich suma również. Stąd będzie istniał co najwyżej jeden argument, dla 
którego zajdzie $ (\frac{3}{5})^x + (\frac{4}{5})^x  = 1 $, a taki już wskazałam.
\end{document}